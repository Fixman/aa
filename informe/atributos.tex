\subsection{Header}

Por cada elemento \(t\) del header de un mensaje (\(date\), \(from\), \(to\), \(cc\), \ldots) agregamos los siguientes atributos:

\begin{enumerate}
  \item{t\_exists}
  \item{t\_length}
  \item{t\_words}
  \item{t\_fields}
  \item{t\_avgWordLength}
  \item{t\_avgFieldLength}
\end{enumerate}

Utilizamos el promedio positivo para calcular los average fields, para evitar una posible excepción \(zeroDivision\): $\frac{1}{1+n} \cdot \sum_{}^{}1+a_{i}$

\subsection{Análisis de texto}

Para obtener atributos relacionados con análisis de texto decidimos trabajar específicamente con 2 campos: \(body\) y \(subject\).

La extracción se realizó en base a aplicar Bayes sobre la probabilidad de que un mensaje sea \spam{} dada la aparición de una palabra (potencial atributo) en dicho mensaje.

Sea $w \in W$, siendo $W$ el universo de palabras obtenido a partir de los mails utilizados para entrenar el modelo.

\vspace{5px}
$P(spam \vert w) = \frac{P(w \vert spam) \cdot P(spam)}{P(w)}$, donde:

\vspace{5px}
$P(w \vert spam) = \frac{\sum_{s \in spam}^{} w \in s}{\mid spam \mid}$, $P(w) = \frac{\sum_{m \in M}^{} w \in m}{\mid M \mid}$, $P(spam) = \frac{\mid spam \mid}{\mid M \mid}$

\vspace{5px}
Análogamente, se realiza la misma operación para analizar las palabras que aparecen en los mails de \ham{}.

El primer experimento consistió en encontrar las 100 palabras que aparezcan en el body de los mails, y que maximicen el valor de Bayes tanto para \spam{} como
para \ham{}. Dicho de otra forma, encontramos las 100 palabras más spammeras y las 100 palabras más hammeras que aparezcan en el \(body\) de los mails.

Posteriormente realizamos un segundo experimento, repitiendo el proceso para las palabras del \(subject\) de los mails.

Durante esta etapa utilizamos la clase \(CountVectorizer\) del módulo \(feature \ extraction\) de \(sklearn\). \(CountVectorizer\) admite como parámetro un token-pattern,
que determina la estructura que deben seguir las palabras analizadas para ser consideradas potenciales features. En este caso, luego de probar con varios valores, optamos
por elegir palabras que contengan solo caracteres entre [a-z], y de longitud mínima = 4. Al mismo tiempo, también admite un parámetro que determina
la cantidad mínima de apariciones que debe tener una palabra para ser considerada \(token\). En este caso, luego de probar con varios valores, determinamos los valores de
800 apariciones para el caso del \(body\) y 200 para el \(subject\). Es decir, solo las palabras que aparecen como mínimo esa cantidad de veces entre la totalidad de
los mails fueron consideradas como tokens.
